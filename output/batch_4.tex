
\documentclass[12pt]{article}
\usepackage[utf8]{inputenc}
\usepackage{amsmath, amssymb}
\usepackage{cancel}
\usepackage{array}
\usepackage{geometry}
\usepackage{fancyhdr}

% Imposta i margini della pagina
\geometry{
    top=2cm,
    bottom=2.5cm,
    left=2.5cm,
    right=2.5cm,
    includehead,
    includefoot
}

% Configura l'header e il footer
\pagestyle{fancy}
\fancyhf{}  % Pulisce header e footer
\renewcommand{\headrulewidth}{0pt}  % Rimuove la linea dell'header
\fancyfoot[C]{-\thepage-}  % Numero di pagina centrato nel footer

% Rimuove lo spazio extra dopo il titolo
\makeatletter
\def\@maketitle{
  \newpage
  \null
  \vskip -2em
  \begin{center}%
  \let \footnote \thanks
    {\LARGE \@title \par}%
    \vskip 1.5em%
    {\large \@date \par}%
  \end{center}%
  \par
  \vskip 1.5em
}
\makeatother

\begin{document}
\title{Soluzione sistema di congruenze}
\date{06/06/2025}
\maketitle
\section*{Esercizio}
Si determinino tutte le soluzioni del seguente sistema di congruenze:
\[
\begin{cases}
x \equiv 4 \pmod{8} \\
x \equiv 12 \pmod{14} \\
\end{cases}
\]
Sia $S \subset \mathbb{Z}$ l'insieme delle soluzioni del sistema.

\textbf{Passo 1: Compatibilità.} \\ 
Grazie al teorema cinese del resto, il sistema è compatibile, cioe $S \neq \emptyset$, se e solo se
$$\gcd(8, 14) \mid 12 - 4 = 8 \qquad (1)$$

Decomponendo in fattori primi:
$$8 = 2^{3}, \quad 14 = 2 \cdot 7$$

Pertanto $\gcd(8, 14) = 2$.

Pertanto la (1) è verificata: $2 \mid 8$. Il sistema è compatibile.\\
Inoltre, osserviamo che $$12 - 4 = 4 \cdot 2 \qquad (2)$$
\textbf{Passo 2: Calcolo di una soluzione particolare} \\
Determiniamo una soluzione $x_0 \in S$.
Iniziamo applicando l'algoritmo di Euclide con sostituzione a ritroso dei resti alla coppia (8, 14):
\begin{center}
\setlength{\arrayrulewidth}{0.5pt}
\begin{tabular}{|p{5cm}|p{9cm}|}
\hline
\textbf{Algoritmo di Euclide} & \textbf{Sostituzione a ritroso} \\
\hline
$8 = 0 \cdot  14 + 8$ & $8 = 8 - 0 \cdot  14$ \\
\hline
$14 = 1 \cdot  8 + 6$ & $6 = 14 - 1 \cdot  8$ \\
\hline
$8 = 1 \cdot  6 + 2$ & $2 = 8 - 1 \cdot  6$ \\
\hline
$\cancel{6 = 3 \cdot  2 + 0}$ & \\
\hline
\end{tabular}
\end{center}
\textbf{Espansione completa:}
\begin{align*}
2 &= 8 - 1\cdot6\\
2 &= 8 - 1\cdot(14 - 1\cdot8)\\
2 &= (8 - 0\cdot14) - 1\cdot(14 - 1\cdot(8 - 0\cdot14))\\
2 &= 2\cdot8 + -1\cdot14\\
\end{align*}
Otteniamo quindi: $$\gcd(8, 14) = 2 = 2 \cdot 8 + -1 \cdot 14$$

Sostituendo nell'equazione (2) otteniamo:
$$12 - 4 = 8 = 2 \cdot 4 = 4 \cdot (2 \cdot 8 + -1 \cdot 14) = -4 \cdot 14 - 8 \cdot 8$$

Ricaviamo adesso una soluzione particolare partendo dalla precedente uguaglianza:
$$12 - 4 = -4 \cdot 14 - 8 \cdot 8 \iff 12 - -4 \cdot 14 = 4 + 8 \cdot 8 \iff 68 = 68$$

Quindi $x_0 = 68 \in S$ è una soluzione particolare del sistema.


\textbf{Passo 3: Calcolo dell'insieme delle soluzioni} \\ 
Grazie al Teorema Cinese del Resto sappiamo che:
$$S = [68]_{\mathrm{mcm}(8, 14)}$$
Ma
$$\mathrm{mcm}({8}, {14}) = \frac{{8} \cdot {14}}{\gcd({8},{14})} = \frac{{2^{3}} \cdot {2 \cdot 7}}{2} = {56}$$
% menomale che nessuno mai vedra' questo codice

Di conseguenza l'insieme delle soluzioni è:
        $$S = [68]_{56} = [68 - 1 \cdot 56]_{56} = [12]_{56} = \{12 + 56 \cdot k : k \in \mathbb{Z}\}$$
        
\end{document}
