
\documentclass[12pt]{article}
\usepackage[utf8]{inputenc}
\usepackage{amsmath, amssymb}
\usepackage{cancel}
\usepackage{array}
\usepackage{geometry}
\usepackage{fancyhdr}

% Imposta i margini della pagina
\geometry{
    top=2cm,
    bottom=2.5cm,
    left=2.5cm,
    right=2.5cm,
    includehead,
    includefoot
}

% Configura l'header e il footer
\pagestyle{fancy}
\fancyhf{}  % Pulisce header e footer
\renewcommand{\headrulewidth}{0pt}  % Rimuove la linea dell'header
\fancyfoot[C]{-\thepage-}  % Numero di pagina centrato nel footer

% Rimuove lo spazio extra dopo il titolo
\makeatletter
\def\@maketitle{
  \newpage
  \null
  \vskip -2em
  \begin{center}%
  \let \footnote \thanks
    {\LARGE \@title \par}%
    \vskip 1.5em%
    {\large \@date \par}%
  \end{center}%
  \par
  \vskip 1.5em
}
\makeatother

\begin{document}
\title{Soluzione sistema di congruenze}
\date{06/06/2025}
\maketitle
\section*{Esercizio}
Si determinino tutte le soluzioni del seguente sistema di congruenze:
\[
\begin{cases}
x \equiv 100 \pmod{150} \\
x \equiv 65 \pmod{85} \\
\end{cases}
\]
Sia $S \subset \mathbb{Z}$ l'insieme delle soluzioni del sistema.

\textbf{Passo 1: Compatibilità.} \\ 
Grazie al teorema cinese del resto, il sistema è compatibile, cioe $S \neq \emptyset$, se e solo se
$$\gcd(150, 85) \mid 65 - 100 = -35 \qquad (1)$$

Decomponendo in fattori primi:
$$150 = 2 \cdot 3 \cdot 5^{2}, \quad 85 = 5 \cdot 17$$

Pertanto $\gcd(150, 85) = 5$.

Pertanto la (1) è verificata: $5 \mid -35$. Il sistema è compatibile.\\
Inoltre, osserviamo che $$65 - 100 = -7 \cdot 5 \qquad (2)$$
\textbf{Passo 2: Calcolo di una soluzione particolare} \\
Determiniamo una soluzione $x_0 \in S$.
Iniziamo applicando l'algoritmo di Euclide con sostituzione a ritroso dei resti alla coppia (150, 85):
\begin{center}
\setlength{\arrayrulewidth}{0.5pt}
\begin{tabular}{|p{5cm}|p{9cm}|}
\hline
\textbf{Algoritmo di Euclide} & \textbf{Sostituzione a ritroso} \\
\hline
$150 = 1 \cdot  85 + 65$ & $65 = 150 - 1 \cdot  85$ \\
\hline
$85 = 1 \cdot  65 + 20$ & $20 = 85 - 1 \cdot  65$ \\
\hline
$65 = 3 \cdot  20 + 5$ & $5 = 65 - 3 \cdot  20$ \\
\hline
$\cancel{20 = 4 \cdot  5 + 0}$ & \\
\hline
\end{tabular}
\end{center}
\textbf{Espansione completa:}
\begin{align*}
5 &= 65 - 3\cdot20\\
5 &= 65 - 3\cdot(85 - 1\cdot65)\\
5 &= (150 - 1\cdot85) - 3\cdot(85 - 1\cdot(150 - 1\cdot85))\\
5 &= 4\cdot150 + -7\cdot85\\
\end{align*}
Otteniamo quindi: $$\gcd(150, 85) = 5 = 4 \cdot 150 + -7 \cdot 85$$

Sostituendo nell'equazione (2) otteniamo:
$$65 - 100 = -35 = 5 \cdot -7 = -7 \cdot (4 \cdot 150 + -7 \cdot 85) = 49 \cdot 85 - 28 \cdot 150$$

Ricaviamo adesso una soluzione particolare partendo dalla precedente uguaglianza:
$$65 - 100 = 49 \cdot 85 - 28 \cdot 150 \iff 65 - 49 \cdot 85 = 100 + 28 \cdot 150 \iff -4100 = -4100$$

Quindi $x_0 = -4100 \in S$ è una soluzione particolare del sistema.


\textbf{Passo 3: Calcolo dell'insieme delle soluzioni} \\ 
Grazie al Teorema Cinese del Resto sappiamo che:
$$S = [-4100]_{\mathrm{mcm}(150, 85)}$$
Ma
$$\mathrm{mcm}({150}, {85}) = \frac{{150} \cdot {85}}{\gcd({150},{85})} = \frac{{2 \cdot 3 \cdot 5^{2}} \cdot {5 \cdot 17}}{5} = {2550}$$
% menomale che nessuno mai vedra' questo codice

Di conseguenza l'insieme delle soluzioni è:
        $$S = [-4100]_{2550} = [-4100 + 2 \cdot 2550]_{2550} = [1000]_{2550} = \{1000 + 2550 \cdot k : k \in \mathbb{Z}\}$$
        
\end{document}
