
\documentclass[12pt]{article}
\usepackage[utf8]{inputenc}
\usepackage{amsmath, amssymb}
\usepackage{cancel}
\usepackage{array}
\usepackage{geometry}
\usepackage{fancyhdr}

% Imposta i margini della pagina
\geometry{
    top=2cm,
    bottom=2.5cm,
    left=2.5cm,
    right=2.5cm,
    includehead,
    includefoot
}

% Configura l'header e il footer
\pagestyle{fancy}
\fancyhf{}  % Pulisce header e footer
\renewcommand{\headrulewidth}{0pt}  % Rimuove la linea dell'header
\fancyfoot[C]{-\thepage-}  % Numero di pagina centrato nel footer

% Rimuove lo spazio extra dopo il titolo
\makeatletter
\def\@maketitle{
  \newpage
  \null
  \vskip -2em
  \begin{center}%
  \let \footnote \thanks
    {\LARGE \@title \par}%
    \vskip 1.5em%
    {\large \@date \par}%
  \end{center}%
  \par
  \vskip 1.5em
}
\makeatother

\begin{document}
\title{Soluzione sistema di congruenze}
\date{06/06/2025}
\maketitle
\section*{Esercizio}
Si determinino tutte le soluzioni del seguente sistema di congruenze:
\[
\begin{cases}
x \equiv 47 \pmod{102} \\
x \equiv 33 \pmod{98} \\
\end{cases}
\]
Sia $S \subset \mathbb{Z}$ l'insieme delle soluzioni del sistema.

\textbf{Passo 1: Compatibilità.} \\ 
Grazie al teorema cinese del resto, il sistema è compatibile, cioe $S \neq \emptyset$, se e solo se
$$\gcd(102, 98) \mid 33 - 47 = -14 \qquad (1)$$

Decomponendo in fattori primi:
$$102 = 2 \cdot 3 \cdot 17, \quad 98 = 2 \cdot 7^{2}$$

Pertanto $\gcd(102, 98) = 2$.

Pertanto la (1) è verificata: $2 \mid -14$. Il sistema è compatibile.\\
Inoltre, osserviamo che $$33 - 47 = -7 \cdot 2 \qquad (2)$$
\textbf{Passo 2: Calcolo di una soluzione particolare} \\
Determiniamo una soluzione $x_0 \in S$.
Iniziamo applicando l'algoritmo di Euclide con sostituzione a ritroso dei resti alla coppia (102, 98):
\begin{center}
\setlength{\arrayrulewidth}{0.5pt}
\begin{tabular}{|p{5cm}|p{9cm}|}
\hline
\textbf{Algoritmo di Euclide} & \textbf{Sostituzione a ritroso} \\
\hline
$102 = 1 \cdot  98 + 4$ & $4 = 102 - 1 \cdot  98$ \\
\hline
$98 = 24 \cdot  4 + 2$ & $2 = 98 - 24 \cdot  4$ \\
\hline
$\cancel{4 = 2 \cdot  2 + 0}$ & \\
\hline
\end{tabular}
\end{center}
\textbf{Espansione completa:}
\begin{align*}
2 &= 98 - 24\cdot4\\
2 &= 98 - 2(102 - 1\cdot98)\cdot(102 - 1\cdot98)\\
2 &= -24\cdot102 + 25\cdot98\\
\end{align*}
Otteniamo quindi: $$\gcd(102, 98) = 2 = -24 \cdot 102 + 25 \cdot 98$$

Sostituendo nell'equazione (2) otteniamo:
$$33 - 47 = -14 = 2 \cdot -7 = -7 \cdot (-24 \cdot 102 + 25 \cdot 98) = -175 \cdot 98 - 168 \cdot 102$$

Ricaviamo adesso una soluzione particolare partendo dalla precedente uguaglianza:
$$33 - 47 = -175 \cdot 98 - 168 \cdot 102 \iff 33 - -175 \cdot 98 = 47 + 168 \cdot 102 \iff 17183 = 17183$$

Quindi $x_0 = 17183 \in S$ è una soluzione particolare del sistema.


\textbf{Passo 3: Calcolo dell'insieme delle soluzioni} \\ 
Grazie al Teorema Cinese del Resto sappiamo che:
$$S = [17183]_{\mathrm{mcm}(102, 98)}$$
Ma
$$\mathrm{mcm}({102}, {98}) = \frac{{102} \cdot {98}}{\gcd({102},{98})} = \frac{{2 \cdot 3 \cdot 17} \cdot {2 \cdot 7^{2}}}{2} = {4998}$$
% menomale che nessuno mai vedra' questo codice

Di conseguenza l'insieme delle soluzioni è:
        $$S = [17183]_{4998} = [17183 - 3 \cdot 4998]_{4998} = [2189]_{4998} = \{2189 + 4998 \cdot k : k \in \mathbb{Z}\}$$
        
\end{document}
