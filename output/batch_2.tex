
\documentclass[12pt]{article}
\usepackage[utf8]{inputenc}
\usepackage{amsmath, amssymb}
\usepackage{cancel}
\usepackage{array}
\usepackage{geometry}
\usepackage{fancyhdr}

% Imposta i margini della pagina
\geometry{
    top=2cm,
    bottom=2.5cm,
    left=2.5cm,
    right=2.5cm,
    includehead,
    includefoot
}

% Configura l'header e il footer
\pagestyle{fancy}
\fancyhf{}  % Pulisce header e footer
\renewcommand{\headrulewidth}{0pt}  % Rimuove la linea dell'header
\fancyfoot[C]{-\thepage-}  % Numero di pagina centrato nel footer

% Rimuove lo spazio extra dopo il titolo
\makeatletter
\def\@maketitle{
  \newpage
  \null
  \vskip -2em
  \begin{center}%
  \let \footnote \thanks
    {\LARGE \@title \par}%
    \vskip 1.5em%
    {\large \@date \par}%
  \end{center}%
  \par
  \vskip 1.5em
}
\makeatother

\begin{document}
\title{Soluzione sistema di congruenze}
\date{06/06/2025}
\maketitle
\section*{Esercizio}
Si determinino tutte le soluzioni del seguente sistema di congruenze:
\[
\begin{cases}
x \equiv 1 \pmod{6} \\
x \equiv 3 \pmod{9} \\
\end{cases}
\]
Sia $S \subset \mathbb{Z}$ l'insieme delle soluzioni del sistema.

\textbf{Passo 1: Compatibilità.} \\ 
Grazie al teorema cinese del resto, il sistema è compatibile, cioe $S \neq \emptyset$, se e solo se
$$\gcd(6, 9) \mid 1 - 3 = -2 \qquad (1)$$

Decomponendo in fattori primi:
$$6 = 2 \cdot 3, \quad 9 = 3^{2}$$

Pertanto $\gcd(6, 9) = 3$.

Pertanto la (1) NON è verificata: $3 \nmid 2$. Il sistema è incompatibile.\\
\end{document}
