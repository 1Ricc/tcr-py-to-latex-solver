
\documentclass[12pt]{article}
\usepackage[utf8]{inputenc}
\usepackage{amsmath, amssymb}
\usepackage{cancel}
\usepackage{array}
\usepackage{geometry}
\usepackage{fancyhdr}

% Imposta i margini della pagina
\geometry{
    top=2cm,
    bottom=2.5cm,
    left=2.5cm,
    right=2.5cm,
    includehead,
    includefoot
}

% Configura l'header e il footer
\pagestyle{fancy}
\fancyhf{}  % Pulisce header e footer
\renewcommand{\headrulewidth}{0pt}  % Rimuove la linea dell'header
\fancyfoot[C]{-\thepage-}  % Numero di pagina centrato nel footer

% Rimuove lo spazio extra dopo il titolo
\makeatletter
\def\@maketitle{
  \newpage
  \null
  \vskip -2em
  \begin{center}%
  \let \footnote \thanks
    {\LARGE \@title \par}%
    \vskip 1.5em%
    {\large \@date \par}%
  \end{center}%
  \par
  \vskip 1.5em
}
\makeatother

\begin{document}
\title{Soluzione sistema di congruenze}
\date{06/06/2025}
\maketitle
\section*{Esercizio}
Si determinino tutte le soluzioni del seguente sistema di congruenze:
\[
\begin{cases}
x \equiv 3 \pmod{1} \\
x \equiv 7 \pmod{4} \\
\end{cases}
\]
Sia $S \subset \mathbb{Z}$ l'insieme delle soluzioni del sistema.

\textbf{Passo 1: Compatibilità.} \\ 
Grazie al teorema cinese del resto, il sistema è compatibile, cioe $S \neq \emptyset$, se e solo se
$$\gcd(1, 4) \mid 7 - 3 = 4 \qquad (1)$$

Decomponendo in fattori primi:
$$1 = 1, \quad 4 = 2^{2}$$

Pertanto $\gcd(1, 4) = 1$.

Pertanto la (1) è verificata: $1 \mid 4$. Il sistema è compatibile.\\
Inoltre, osserviamo che $$7 - 3 = 4 \cdot 1 \qquad (2)$$
\textbf{Passo 2: Calcolo di una soluzione particolare} \\
Determiniamo una soluzione $x_0 \in S$.
Iniziamo applicando l'algoritmo di Euclide con sostituzione a ritroso dei resti alla coppia (1, 4):
\begin{center}
\setlength{\arrayrulewidth}{0.5pt}
\begin{tabular}{|p{5cm}|p{9cm}|}
\hline
\textbf{Algoritmo di Euclide} & \textbf{Sostituzione a ritroso} \\
\hline
$1 = 0 \cdot  4 + 1$ & $1 = 1 - 0 \cdot  4$ \\
\hline
$\cancel{4 = 4 \cdot  1 + 0}$ & \\
\hline
\end{tabular}
\end{center}
\textbf{Espansione completa:}
\begin{align*}
1 &= 1 - 0\cdot4\\
1 &= 1\cdot1 + 0\cdot4\\
\end{align*}
Otteniamo quindi: $$\gcd(1, 4) = 1 = 1 \cdot 1 + 0 \cdot 4$$

Sostituendo nell'equazione (2) otteniamo:
$$7 - 3 = 4 = 1 \cdot 4 = 4 \cdot (1 \cdot 1 + 0 \cdot 4) = 0 \cdot 4 - 4 \cdot 1$$

Ricaviamo adesso una soluzione particolare partendo dalla precedente uguaglianza:
$$7 - 3 = 0 \cdot 4 - 4 \cdot 1 \iff 7 - 0 \cdot 4 = 3 + 4 \cdot 1 \iff 7 = 7$$

Quindi $x_0 = 7 \in S$ è una soluzione particolare del sistema.


\textbf{Passo 3: Calcolo dell'insieme delle soluzioni} \\ 
Grazie al Teorema Cinese del Resto sappiamo che:
$$S = [7]_{\mathrm{mcm}(1, 4)}$$
Ma
$$\mathrm{mcm}({1}, {4}) = \frac{{1} \cdot {4}}{\gcd({1},{4})} = \frac{{1} \cdot {2^{2}}}{1} = {4}$$
% menomale che nessuno mai vedra' questo codice

Di conseguenza l'insieme delle soluzioni è:
        $$S = [7]_{4} = [7 - 1 \cdot 4]_{4} = [3]_{4} = \{3 + 4 \cdot k : k \in \mathbb{Z}\}$$
        
\end{document}
