
\documentclass[12pt]{article}
\usepackage[utf8]{inputenc}
\usepackage{amsmath, amssymb}
\usepackage{cancel}
\usepackage{array}
\usepackage{geometry}
\usepackage{fancyhdr}

% Imposta i margini della pagina
\geometry{
    top=2cm,
    bottom=2.5cm,
    left=2.5cm,
    right=2.5cm,
    includehead,
    includefoot
}

% Configura l'header e il footer
\pagestyle{fancy}
\fancyhf{}  % Pulisce header e footer
\renewcommand{\headrulewidth}{0pt}  % Rimuove la linea dell'header
\fancyfoot[C]{-\thepage-}  % Numero di pagina centrato nel footer

% Rimuove lo spazio extra dopo il titolo
\makeatletter
\def\@maketitle{
  \newpage
  \null
  \vskip -2em
  \begin{center}%
  \let \footnote \thanks
    {\LARGE \@title \par}%
    \vskip 1.5em%
    {\large \@date \par}%
  \end{center}%
  \par
  \vskip 1.5em
}
\makeatother

\begin{document}
\title{Soluzione sistema di congruenze}
\date{06/06/2025}
\maketitle
\section*{Esercizio}
Si determinino tutte le soluzioni del seguente sistema di congruenze:
\[
\begin{cases}
x \equiv 2 \pmod{3} \\
x \equiv 3 \pmod{5} \\
\end{cases}
\]
Sia $S \subset \mathbb{Z}$ l'insieme delle soluzioni del sistema.

\textbf{Passo 1: Compatibilità.} \\ 
Grazie al teorema cinese del resto, il sistema è compatibile, cioe $S \neq \emptyset$, se e solo se
$$\gcd(3, 5) \mid 3 - 2 = 1 \qquad (1)$$

Decomponendo in fattori primi:
$$3 = 3, \quad 5 = 5$$

Pertanto $\gcd(3, 5) = 1$.

Pertanto la (1) è verificata: $1 \mid 1$. Il sistema è compatibile.\\
Inoltre, osserviamo che $$3 - 2 = 1 \cdot 1 \qquad (2)$$
\textbf{Passo 2: Calcolo di una soluzione particolare} \\
Determiniamo una soluzione $x_0 \in S$.
Iniziamo applicando l'algoritmo di Euclide con sostituzione a ritroso dei resti alla coppia (3, 5):
\begin{center}
\setlength{\arrayrulewidth}{0.5pt}
\begin{tabular}{|p{5cm}|p{9cm}|}
\hline
\textbf{Algoritmo di Euclide} & \textbf{Sostituzione a ritroso} \\
\hline
$3 = 0 \cdot  5 + 3$ & $3 = 3 - 0 \cdot  5$ \\
\hline
$5 = 1 \cdot  3 + 2$ & $2 = 5 - 1 \cdot  3$ \\
\hline
$3 = 1 \cdot  2 + 1$ & $1 = 3 - 1 \cdot  2$ \\
\hline
$\cancel{2 = 2 \cdot  1 + 0}$ & \\
\hline
\end{tabular}
\end{center}
\textbf{Espansione completa:}
\begin{align*}
1 &= 3 - 1\cdot2\\
1 &= 3 - 1\cdot(5 - 1\cdot3)\\
1 &= (3 - 0\cdot5) - 1\cdot(5 - 1\cdot(3 - 0\cdot5))\\
1 &= 2\cdot3 + -1\cdot5\\
\end{align*}
Otteniamo quindi: $$\gcd(3, 5) = 1 = 2 \cdot 3 + -1 \cdot 5$$

Sostituendo nell'equazione (2) otteniamo:
$$3 - 2 = 1 = 1 \cdot 1 = 1 \cdot (2 \cdot 3 + -1 \cdot 5) = -1 \cdot 5 - 2 \cdot 3$$

Ricaviamo adesso una soluzione particolare partendo dalla precedente uguaglianza:
$$3 - 2 = -1 \cdot 5 - 2 \cdot 3 \iff 3 - -1 \cdot 5 = 2 + 2 \cdot 3 \iff 8 = 8$$

Quindi $x_0 = 8 \in S$ è una soluzione particolare del sistema.


\textbf{Passo 3: Calcolo dell'insieme delle soluzioni} \\ 
Grazie al Teorema Cinese del Resto sappiamo che:
$$S = [8]_{\mathrm{mcm}(3, 5)}$$
Ma
$$\mathrm{mcm}({3}, {5}) = \frac{{3} \cdot {5}}{\gcd({3},{5})} = \frac{{3} \cdot {5}}{1} = {15}$$
% menomale che nessuno mai vedra' questo codice

Di conseguenza l'insieme delle soluzioni è:
        $$S = [8]_{15} = \{8 + 15 \cdot k : k \in \mathbb{Z}\}$$
        
\end{document}
