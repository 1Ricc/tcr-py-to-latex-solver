
\documentclass[12pt]{article}
\usepackage[utf8]{inputenc}
\usepackage{amsmath, amssymb}
\usepackage{cancel}
\usepackage{array}
\usepackage{geometry}
\usepackage{fancyhdr}

% Imposta i margini della pagina
\geometry{
    top=2cm,
    bottom=2.5cm,
    left=2.5cm,
    right=2.5cm,
    includehead,
    includefoot
}

% Configura l'header e il footer
\pagestyle{fancy}
\fancyhf{}  % Pulisce header e footer
\renewcommand{\headrulewidth}{0pt}  % Rimuove la linea dell'header
\fancyfoot[C]{-\thepage-}  % Numero di pagina centrato nel footer

% Rimuove lo spazio extra dopo il titolo
\makeatletter
\def\@maketitle{
  \newpage
  \null
  \vskip -2em
  \begin{center}%
  \let \footnote \thanks
    {\LARGE \@title \par}%
    \vskip 1.5em%
    {\large \@date \par}%
  \end{center}%
  \par
  \vskip 1.5em
}
\makeatother

\begin{document}
\title{Soluzione sistema di congruenze}
\date{06/06/2025}
\maketitle
\section*{Esercizio}
Si determinino tutte le soluzioni del seguente sistema di congruenze:
\[
\begin{cases}
x \equiv 28 \pmod{108} \\
x \equiv 64 \pmod{78} \\
\end{cases}
\]
Sia $S \subset \mathbb{Z}$ l'insieme delle soluzioni del sistema.

\textbf{Passo 1: Compatibilità.} \\ 
Grazie al teorema cinese del resto, il sistema è compatibile, cioe $S \neq \emptyset$, se e solo se
$$\gcd(108, 78) \mid 64 - 28 = 36 \qquad (1)$$

Decomponendo in fattori primi:
$$108 = 2^{2} \cdot 3^{3}, \quad 78 = 2 \cdot 3 \cdot 13$$

Pertanto $\gcd(108, 78) = 6$.

Pertanto la (1) è verificata: $6 \mid 36$. Il sistema è compatibile.\\
Inoltre, osserviamo che $$64 - 28 = 6 \cdot 6 \qquad (2)$$
\textbf{Passo 2: Calcolo di una soluzione particolare} \\
Determiniamo una soluzione $x_0 \in S$.
Iniziamo applicando l'algoritmo di Euclide con sostituzione a ritroso dei resti alla coppia (108, 78):
\begin{center}
\setlength{\arrayrulewidth}{0.5pt}
\begin{tabular}{|p{5cm}|p{9cm}|}
\hline
\textbf{Algoritmo di Euclide} & \textbf{Sostituzione a ritroso} \\
\hline
$108 = 1 \cdot  78 + 30$ & $30 = 108 - 1 \cdot  78$ \\
\hline
$78 = 2 \cdot  30 + 18$ & $18 = 78 - 2 \cdot  30$ \\
\hline
$30 = 1 \cdot  18 + 12$ & $12 = 30 - 1 \cdot  18$ \\
\hline
$18 = 1 \cdot  12 + 6$ & $6 = 18 - 1 \cdot  12$ \\
\hline
$\cancel{12 = 2 \cdot  6 + 0}$ & \\
\hline
\end{tabular}
\end{center}
\textbf{Espansione completa:}
\begin{align*}
6 &= 18 - 1\cdot12\\
6 &= 18 - 1\cdot(30 - 1\cdot18)\\
6 &= (78 - 2\cdot30) - 1\cdot(30 - 1\cdot(78 - 2\cdot30))\\
6 &= (78 - 2\cdot(108 - 1\cdot78)) - 1\cdot((108 - 1\cdot78) - 1\cdot(78 - 2\cdot(108 - 1\cdot78)))\\
6 &= -5\cdot108 + 7\cdot78\\
\end{align*}
Otteniamo quindi: $$\gcd(108, 78) = 6 = -5 \cdot 108 + 7 \cdot 78$$

Sostituendo nell'equazione (2) otteniamo:
$$64 - 28 = 36 = 6 \cdot 6 = 6 \cdot (-5 \cdot 108 + 7 \cdot 78) = 42 \cdot 78 - 30 \cdot 108$$

Ricaviamo adesso una soluzione particolare partendo dalla precedente uguaglianza:
$$64 - 28 = 42 \cdot 78 - 30 \cdot 108 \iff 64 - 42 \cdot 78 = 28 + 30 \cdot 108 \iff -3212 = -3212$$

Quindi $x_0 = -3212 \in S$ è una soluzione particolare del sistema.


\textbf{Passo 3: Calcolo dell'insieme delle soluzioni} \\ 
Grazie al Teorema Cinese del Resto sappiamo che:
$$S = [-3212]_{\mathrm{mcm}(108, 78)}$$
Ma
$$\mathrm{mcm}({108}, {78}) = \frac{{108} \cdot {78}}{\gcd({108},{78})} = \frac{{2^{2} \cdot 3^{3}} \cdot {2 \cdot 3 \cdot 13}}{6} = {1404}$$
% menomale che nessuno mai vedra' questo codice

Di conseguenza l'insieme delle soluzioni è:
        $$S = [-3212]_{1404} = [-3212 + 3 \cdot 1404]_{1404} = [1000]_{1404} = \{1000 + 1404 \cdot k : k \in \mathbb{Z}\}$$
        
\end{document}
